% Generated by Sphinx.
\def\sphinxdocclass{report}
\documentclass[letterpaper,10pt,english]{sphinxmanual}
\usepackage[utf8]{inputenc}
\DeclareUnicodeCharacter{00A0}{\nobreakspace}
\usepackage{cmap}
\usepackage[T1]{fontenc}
\usepackage{babel}
\usepackage{times}
\usepackage[Bjarne]{fncychap}
\usepackage{longtable}
\usepackage{sphinx}
\usepackage{multirow}


\title{python-shop-sdk Documentation}
\date{March 26, 2014}
\release{0.1}
\author{Arne Simon =\textgreater{} email::{[}arne.simon@slice-dice.de{]}}
\newcommand{\sphinxlogo}{}
\renewcommand{\releasename}{Release}
\makeindex

\makeatletter
\def\PYG@reset{\let\PYG@it=\relax \let\PYG@bf=\relax%
    \let\PYG@ul=\relax \let\PYG@tc=\relax%
    \let\PYG@bc=\relax \let\PYG@ff=\relax}
\def\PYG@tok#1{\csname PYG@tok@#1\endcsname}
\def\PYG@toks#1+{\ifx\relax#1\empty\else%
    \PYG@tok{#1}\expandafter\PYG@toks\fi}
\def\PYG@do#1{\PYG@bc{\PYG@tc{\PYG@ul{%
    \PYG@it{\PYG@bf{\PYG@ff{#1}}}}}}}
\def\PYG#1#2{\PYG@reset\PYG@toks#1+\relax+\PYG@do{#2}}

\expandafter\def\csname PYG@tok@gd\endcsname{\def\PYG@tc##1{\textcolor[rgb]{0.63,0.00,0.00}{##1}}}
\expandafter\def\csname PYG@tok@gu\endcsname{\let\PYG@bf=\textbf\def\PYG@tc##1{\textcolor[rgb]{0.50,0.00,0.50}{##1}}}
\expandafter\def\csname PYG@tok@gt\endcsname{\def\PYG@tc##1{\textcolor[rgb]{0.00,0.27,0.87}{##1}}}
\expandafter\def\csname PYG@tok@gs\endcsname{\let\PYG@bf=\textbf}
\expandafter\def\csname PYG@tok@gr\endcsname{\def\PYG@tc##1{\textcolor[rgb]{1.00,0.00,0.00}{##1}}}
\expandafter\def\csname PYG@tok@cm\endcsname{\let\PYG@it=\textit\def\PYG@tc##1{\textcolor[rgb]{0.25,0.50,0.56}{##1}}}
\expandafter\def\csname PYG@tok@vg\endcsname{\def\PYG@tc##1{\textcolor[rgb]{0.73,0.38,0.84}{##1}}}
\expandafter\def\csname PYG@tok@m\endcsname{\def\PYG@tc##1{\textcolor[rgb]{0.13,0.50,0.31}{##1}}}
\expandafter\def\csname PYG@tok@mh\endcsname{\def\PYG@tc##1{\textcolor[rgb]{0.13,0.50,0.31}{##1}}}
\expandafter\def\csname PYG@tok@cs\endcsname{\def\PYG@tc##1{\textcolor[rgb]{0.25,0.50,0.56}{##1}}\def\PYG@bc##1{\setlength{\fboxsep}{0pt}\colorbox[rgb]{1.00,0.94,0.94}{\strut ##1}}}
\expandafter\def\csname PYG@tok@ge\endcsname{\let\PYG@it=\textit}
\expandafter\def\csname PYG@tok@vc\endcsname{\def\PYG@tc##1{\textcolor[rgb]{0.73,0.38,0.84}{##1}}}
\expandafter\def\csname PYG@tok@il\endcsname{\def\PYG@tc##1{\textcolor[rgb]{0.13,0.50,0.31}{##1}}}
\expandafter\def\csname PYG@tok@go\endcsname{\def\PYG@tc##1{\textcolor[rgb]{0.20,0.20,0.20}{##1}}}
\expandafter\def\csname PYG@tok@cp\endcsname{\def\PYG@tc##1{\textcolor[rgb]{0.00,0.44,0.13}{##1}}}
\expandafter\def\csname PYG@tok@gi\endcsname{\def\PYG@tc##1{\textcolor[rgb]{0.00,0.63,0.00}{##1}}}
\expandafter\def\csname PYG@tok@gh\endcsname{\let\PYG@bf=\textbf\def\PYG@tc##1{\textcolor[rgb]{0.00,0.00,0.50}{##1}}}
\expandafter\def\csname PYG@tok@ni\endcsname{\let\PYG@bf=\textbf\def\PYG@tc##1{\textcolor[rgb]{0.84,0.33,0.22}{##1}}}
\expandafter\def\csname PYG@tok@nl\endcsname{\let\PYG@bf=\textbf\def\PYG@tc##1{\textcolor[rgb]{0.00,0.13,0.44}{##1}}}
\expandafter\def\csname PYG@tok@nn\endcsname{\let\PYG@bf=\textbf\def\PYG@tc##1{\textcolor[rgb]{0.05,0.52,0.71}{##1}}}
\expandafter\def\csname PYG@tok@no\endcsname{\def\PYG@tc##1{\textcolor[rgb]{0.38,0.68,0.84}{##1}}}
\expandafter\def\csname PYG@tok@na\endcsname{\def\PYG@tc##1{\textcolor[rgb]{0.25,0.44,0.63}{##1}}}
\expandafter\def\csname PYG@tok@nb\endcsname{\def\PYG@tc##1{\textcolor[rgb]{0.00,0.44,0.13}{##1}}}
\expandafter\def\csname PYG@tok@nc\endcsname{\let\PYG@bf=\textbf\def\PYG@tc##1{\textcolor[rgb]{0.05,0.52,0.71}{##1}}}
\expandafter\def\csname PYG@tok@nd\endcsname{\let\PYG@bf=\textbf\def\PYG@tc##1{\textcolor[rgb]{0.33,0.33,0.33}{##1}}}
\expandafter\def\csname PYG@tok@ne\endcsname{\def\PYG@tc##1{\textcolor[rgb]{0.00,0.44,0.13}{##1}}}
\expandafter\def\csname PYG@tok@nf\endcsname{\def\PYG@tc##1{\textcolor[rgb]{0.02,0.16,0.49}{##1}}}
\expandafter\def\csname PYG@tok@si\endcsname{\let\PYG@it=\textit\def\PYG@tc##1{\textcolor[rgb]{0.44,0.63,0.82}{##1}}}
\expandafter\def\csname PYG@tok@s2\endcsname{\def\PYG@tc##1{\textcolor[rgb]{0.25,0.44,0.63}{##1}}}
\expandafter\def\csname PYG@tok@vi\endcsname{\def\PYG@tc##1{\textcolor[rgb]{0.73,0.38,0.84}{##1}}}
\expandafter\def\csname PYG@tok@nt\endcsname{\let\PYG@bf=\textbf\def\PYG@tc##1{\textcolor[rgb]{0.02,0.16,0.45}{##1}}}
\expandafter\def\csname PYG@tok@nv\endcsname{\def\PYG@tc##1{\textcolor[rgb]{0.73,0.38,0.84}{##1}}}
\expandafter\def\csname PYG@tok@s1\endcsname{\def\PYG@tc##1{\textcolor[rgb]{0.25,0.44,0.63}{##1}}}
\expandafter\def\csname PYG@tok@gp\endcsname{\let\PYG@bf=\textbf\def\PYG@tc##1{\textcolor[rgb]{0.78,0.36,0.04}{##1}}}
\expandafter\def\csname PYG@tok@sh\endcsname{\def\PYG@tc##1{\textcolor[rgb]{0.25,0.44,0.63}{##1}}}
\expandafter\def\csname PYG@tok@ow\endcsname{\let\PYG@bf=\textbf\def\PYG@tc##1{\textcolor[rgb]{0.00,0.44,0.13}{##1}}}
\expandafter\def\csname PYG@tok@sx\endcsname{\def\PYG@tc##1{\textcolor[rgb]{0.78,0.36,0.04}{##1}}}
\expandafter\def\csname PYG@tok@bp\endcsname{\def\PYG@tc##1{\textcolor[rgb]{0.00,0.44,0.13}{##1}}}
\expandafter\def\csname PYG@tok@c1\endcsname{\let\PYG@it=\textit\def\PYG@tc##1{\textcolor[rgb]{0.25,0.50,0.56}{##1}}}
\expandafter\def\csname PYG@tok@kc\endcsname{\let\PYG@bf=\textbf\def\PYG@tc##1{\textcolor[rgb]{0.00,0.44,0.13}{##1}}}
\expandafter\def\csname PYG@tok@c\endcsname{\let\PYG@it=\textit\def\PYG@tc##1{\textcolor[rgb]{0.25,0.50,0.56}{##1}}}
\expandafter\def\csname PYG@tok@mf\endcsname{\def\PYG@tc##1{\textcolor[rgb]{0.13,0.50,0.31}{##1}}}
\expandafter\def\csname PYG@tok@err\endcsname{\def\PYG@bc##1{\setlength{\fboxsep}{0pt}\fcolorbox[rgb]{1.00,0.00,0.00}{1,1,1}{\strut ##1}}}
\expandafter\def\csname PYG@tok@kd\endcsname{\let\PYG@bf=\textbf\def\PYG@tc##1{\textcolor[rgb]{0.00,0.44,0.13}{##1}}}
\expandafter\def\csname PYG@tok@ss\endcsname{\def\PYG@tc##1{\textcolor[rgb]{0.32,0.47,0.09}{##1}}}
\expandafter\def\csname PYG@tok@sr\endcsname{\def\PYG@tc##1{\textcolor[rgb]{0.14,0.33,0.53}{##1}}}
\expandafter\def\csname PYG@tok@mo\endcsname{\def\PYG@tc##1{\textcolor[rgb]{0.13,0.50,0.31}{##1}}}
\expandafter\def\csname PYG@tok@mi\endcsname{\def\PYG@tc##1{\textcolor[rgb]{0.13,0.50,0.31}{##1}}}
\expandafter\def\csname PYG@tok@kn\endcsname{\let\PYG@bf=\textbf\def\PYG@tc##1{\textcolor[rgb]{0.00,0.44,0.13}{##1}}}
\expandafter\def\csname PYG@tok@o\endcsname{\def\PYG@tc##1{\textcolor[rgb]{0.40,0.40,0.40}{##1}}}
\expandafter\def\csname PYG@tok@kr\endcsname{\let\PYG@bf=\textbf\def\PYG@tc##1{\textcolor[rgb]{0.00,0.44,0.13}{##1}}}
\expandafter\def\csname PYG@tok@s\endcsname{\def\PYG@tc##1{\textcolor[rgb]{0.25,0.44,0.63}{##1}}}
\expandafter\def\csname PYG@tok@kp\endcsname{\def\PYG@tc##1{\textcolor[rgb]{0.00,0.44,0.13}{##1}}}
\expandafter\def\csname PYG@tok@w\endcsname{\def\PYG@tc##1{\textcolor[rgb]{0.73,0.73,0.73}{##1}}}
\expandafter\def\csname PYG@tok@kt\endcsname{\def\PYG@tc##1{\textcolor[rgb]{0.56,0.13,0.00}{##1}}}
\expandafter\def\csname PYG@tok@sc\endcsname{\def\PYG@tc##1{\textcolor[rgb]{0.25,0.44,0.63}{##1}}}
\expandafter\def\csname PYG@tok@sb\endcsname{\def\PYG@tc##1{\textcolor[rgb]{0.25,0.44,0.63}{##1}}}
\expandafter\def\csname PYG@tok@k\endcsname{\let\PYG@bf=\textbf\def\PYG@tc##1{\textcolor[rgb]{0.00,0.44,0.13}{##1}}}
\expandafter\def\csname PYG@tok@se\endcsname{\let\PYG@bf=\textbf\def\PYG@tc##1{\textcolor[rgb]{0.25,0.44,0.63}{##1}}}
\expandafter\def\csname PYG@tok@sd\endcsname{\let\PYG@it=\textit\def\PYG@tc##1{\textcolor[rgb]{0.25,0.44,0.63}{##1}}}

\def\PYGZbs{\char`\\}
\def\PYGZus{\char`\_}
\def\PYGZob{\char`\{}
\def\PYGZcb{\char`\}}
\def\PYGZca{\char`\^}
\def\PYGZam{\char`\&}
\def\PYGZlt{\char`\<}
\def\PYGZgt{\char`\>}
\def\PYGZsh{\char`\#}
\def\PYGZpc{\char`\%}
\def\PYGZdl{\char`\$}
\def\PYGZhy{\char`\-}
\def\PYGZsq{\char`\'}
\def\PYGZdq{\char`\"}
\def\PYGZti{\char`\~}
% for compatibility with earlier versions
\def\PYGZat{@}
\def\PYGZlb{[}
\def\PYGZrb{]}
\makeatother

\begin{document}

\maketitle
\tableofcontents
\phantomsection\label{index::doc}


Contents:


\chapter{collins}
\label{collins:collins}\label{collins:welcome-to-python-shop-sdk-s-documentation}\label{collins::doc}\label{collins:module-collins}\index{collins (module)}\begin{quote}\begin{description}
\item[{Author}] \leavevmode
Arne Simon =\textgreater{} email::{[}arne\_simon@gmx.de{]}

\end{description}\end{quote}
\index{COLLINS\_VERSION (in module collins)}

\begin{fulllineitems}
\phantomsection\label{collins:collins.COLLINS_VERSION}\pysigline{\code{collins.}\bfcode{COLLINS\_VERSION}\strong{ = `1.1'}}
The version of the collins api which is supported.

\end{fulllineitems}

\index{Collins (class in collins)}

\begin{fulllineitems}
\phantomsection\label{collins:collins.Collins}\pysiglinewithargsret{\strong{class }\code{collins.}\bfcode{Collins}}{\emph{config}}{}
Bases: \code{object}

An interface to the Collins API.

This is thin warper around the Collins API.
All functions return the JSON responses as Python List and Dictonarys.
\index{autocomplete() (collins.Collins method)}

\begin{fulllineitems}
\phantomsection\label{collins:collins.Collins.autocomplete}\pysiglinewithargsret{\bfcode{autocomplete}}{\emph{text, limit=10, categories={[}'categories', `products'{]}}}{}~\begin{quote}\begin{description}
\item[{Parameters}] \leavevmode
\textbf{text} (\emph{str}) -- The abbriviation.

\end{description}\end{quote}

\begin{notice}{attention}{Attention:}
In the documentation stands \textbf{autocomplete} but the real
Tag is \textbf{autocompletion}!
\end{notice}

\end{fulllineitems}

\index{basketadd() (collins.Collins method)}

\begin{fulllineitems}
\phantomsection\label{collins:collins.Collins.basketadd}\pysiglinewithargsret{\bfcode{basketadd}}{\emph{products}}{}
\end{fulllineitems}

\index{basketget() (collins.Collins method)}

\begin{fulllineitems}
\phantomsection\label{collins:collins.Collins.basketget}\pysiglinewithargsret{\bfcode{basketget}}{\emph{sessionid}}{}
\end{fulllineitems}

\index{category() (collins.Collins method)}

\begin{fulllineitems}
\phantomsection\label{collins:collins.Collins.category}\pysiglinewithargsret{\bfcode{category}}{\emph{ids}}{}
You are able to retrieve single categories.
\begin{quote}\begin{description}
\item[{Parameters}] \leavevmode
\textbf{ids} (\emph{list}) -- List of category ids.

\end{description}\end{quote}

\end{fulllineitems}

\index{categorytree() (collins.Collins method)}

\begin{fulllineitems}
\phantomsection\label{collins:collins.Collins.categorytree}\pysiglinewithargsret{\bfcode{categorytree}}{\emph{max\_depth=None}}{}
The request category tree returns a tree of categories of a 
specified max depth for your app id.

\end{fulllineitems}

\index{facets() (collins.Collins method)}

\begin{fulllineitems}
\phantomsection\label{collins:collins.Collins.facets}\pysiglinewithargsret{\bfcode{facets}}{\emph{group\_ids=None}, \emph{limit=None}, \emph{offset=0}}{}
This returns a list of available facet groups or a facets of a group.
\begin{quote}\begin{description}
\item[{Parameters}] \leavevmode\begin{itemize}
\item {} 
\textbf{group\_ids} (\emph{list}) -- 

\item {} 
\textbf{limit} (\emph{int}) -- 

\item {} 
\textbf{offset} (\emph{int}) -- 

\end{itemize}

\end{description}\end{quote}

\end{fulllineitems}

\index{facettypes() (collins.Collins method)}

\begin{fulllineitems}
\phantomsection\label{collins:collins.Collins.facettypes}\pysiglinewithargsret{\bfcode{facettypes}}{}{}
This query returns a list of facet groups available.

\end{fulllineitems}

\index{getorder() (collins.Collins method)}

\begin{fulllineitems}
\phantomsection\label{collins:collins.Collins.getorder}\pysiglinewithargsret{\bfcode{getorder}}{\emph{orderid}}{}~\begin{description}
\item[{Through this query you could get a order which was created }] \leavevmode
for/through your app. This is limited to a configured 
timeframe and to your app.

\end{description}
\begin{quote}\begin{description}
\item[{Parameters}] \leavevmode
\textbf{orderid} (\emph{int}) -- 

\end{description}\end{quote}

\end{fulllineitems}

\index{initiateorder() (collins.Collins method)}

\begin{fulllineitems}
\phantomsection\label{collins:collins.Collins.initiateorder}\pysiglinewithargsret{\bfcode{initiateorder}}{\emph{sessionid}, \emph{sucess\_url}, \emph{cancel\_url=None}, \emph{error\_url=None}}{}~\begin{description}
\item[{At this request you initiate a order to a basket.}] \leavevmode
This should be done if a user wants to go to the checkout.

\end{description}
\begin{quote}\begin{description}
\item[{Parameters}] \leavevmode\begin{itemize}
\item {} 
\textbf{sessionid} (\emph{str}) -- identification of the basket -\textgreater{} user, user -\textgreater{} basket (see basket\_get, basket\_add)

\item {} 
\textbf{sucess\_url} (\emph{str}) -- this is a callback url if the order was successfully created. (see checkout api)

\item {} 
\textbf{cancel\_url} (\emph{str}) -- this is a callback url if the order was canceled. (see checkout api)

\item {} 
\textbf{error\_url} (\emph{str}) -- this is a callback url if the order throwed exceptions (see checkout api)

\end{itemize}

\end{description}\end{quote}

\end{fulllineitems}

\index{livevariant() (collins.Collins method)}

\begin{fulllineitems}
\phantomsection\label{collins:collins.Collins.livevariant}\pysiglinewithargsret{\bfcode{livevariant}}{\emph{ids}}{}~\begin{description}
\item[{This does return the live information about the product variant. }] \leavevmode
This is as ``live'' as possible.
And could differ vs. a ``product search'' or ``product'' query.

\end{description}
\begin{quote}\begin{description}
\item[{Parameters}] \leavevmode
\textbf{ids} (\emph{list}) -- array of product variant id

\end{description}\end{quote}

\end{fulllineitems}

\index{products() (collins.Collins method)}

\begin{fulllineitems}
\phantomsection\label{collins:collins.Collins.products}\pysiglinewithargsret{\bfcode{products}}{}{}~\begin{description}
\item[{Here you get a detail view of a product or a list of products }] \leavevmode
returned by its ids.

\end{description}
\begin{quote}\begin{description}
\item[{Parameters}] \leavevmode
\textbf{ids} (\emph{list}) -- array of product variant id

\end{description}\end{quote}

\end{fulllineitems}

\index{productsearch() (collins.Collins method)}

\begin{fulllineitems}
\phantomsection\label{collins:collins.Collins.productsearch}\pysiglinewithargsret{\bfcode{productsearch}}{\emph{sessionid}, \emph{filter=None}, \emph{result=None}}{}
This is the main query for retrieving products for your app.
Lists of products will be returned which are filtered.
In a search you dont get back inactive set products.
The response can contain the available facets and their product count and 
categories for the filter set.
\begin{description}
\item[{There are two types of facets:}] \leavevmode\begin{enumerate}
\item {} \begin{description}
\item[{is a pre defined and always available set of facets}] \leavevmode\begin{enumerate}
\item {} 
price range (amazon like see : prices field in response schema)

\item {} 
sale facet how many products are sale and how many not

\item {} 
categories

\end{enumerate}

\end{description}

\item {} 
dynamic facets which occur through the products itself. If a product has one the these attributes you are able to
query for these attributes. These facets are always just ids. The name -\textgreater{} id, id -\textgreater{} name resolution
works through facets call. To get all the facet groups you can filter or are in your result set see ``Facet types''

\end{enumerate}

\end{description}
\begin{quote}\begin{description}
\item[{Parameters}] \leavevmode\begin{itemize}
\item {} 
\textbf{sessionid} (\emph{str}) -- the session\_id of the frontend customer

\item {} 
\textbf{filter} (\emph{dict}) -- object of filter information, these filters do 
change the subset of products

\item {} 
\textbf{result} (\emph{dict}) -- object of result information, these properties 
change the order and the appearance of the subset

\end{itemize}

\end{description}\end{quote}
\paragraph{filter dict}

\begin{tabulary}{\linewidth}{|L|L|}
\hline
\textsf{\relax 
fieldname
} & \textsf{\relax 
meaning
}\\
\hline
categories
 & 
array of category ids to include the products from
\\

sale
 & 
true =\textgreater{} only sale products
false =\textgreater{} no sale products
null =\textgreater{} both
\\

prices
 & 
is an object with properties from and to which lead to
range filter on the price set as eurocent
\\

searchword
facets
 & 
the word which filters every product
object of attributes every key is an array of facet id
values to filter by them
\\
\hline\end{tabulary}

\paragraph{result dict}

\begin{tabulary}{\linewidth}{|L|L|}
\hline
\textsf{\relax 
fieldname
} & \textsf{\relax 
meaning
}\\
\hline
sort
 & 
how to sort the products
\\

price
 & 
geta back a range of attributes and their product counts
also some statistical information in eurocent.
\\

sale
 & 
product count of products which are in sale
\\

facets
 & 
object of which facet get back what count of facets
\\

categories
 & 
set if get or not back the categories in which are products
and the count of products
\\

limit
 & 
how many products to get back
\\

offset
 & 
the offset where to start to get the products from
\\
\hline\end{tabulary}

\paragraph{result.facets dict}

\begin{tabulary}{\linewidth}{|L|L|}
\hline
\textsf{\relax 
fieldname
} & \textsf{\relax 
meaning
}\\
\hline
\_all
 & 
is an object with property limit which holds an integer to
get the amount of attributes, sort by occurrences in products
if this is set, this is the default for ALL attributes.
\\

{[} 0-9 {]}+
 & 
is an object with property limit which holds an integer to
get the amount of attributes of this attribute group id,
sort by occurrences in products. this overwrites even the
\_all field.
\\
\hline\end{tabulary}


\end{fulllineitems}

\index{send() (collins.Collins method)}

\begin{fulllineitems}
\phantomsection\label{collins:collins.Collins.send}\pysiglinewithargsret{\bfcode{send}}{\emph{cmd}}{}
Sends a Pyhton structure of dict's and list's as raw JSON to collins and 
returns a Python structure of dict's and list's from the JSON answer.
\begin{quote}\begin{description}
\item[{Parameters}] \leavevmode
\textbf{cmd} -- The list or dict object to send.

\end{description}\end{quote}

\end{fulllineitems}


\end{fulllineitems}

\index{CollinsException}

\begin{fulllineitems}
\phantomsection\label{collins:collins.CollinsException}\pysigline{\strong{exception }\code{collins.}\bfcode{CollinsException}}
Bases: \code{exceptions.Exception}

An exception in collins module.

\end{fulllineitems}

\index{Config (class in collins)}

\begin{fulllineitems}
\phantomsection\label{collins:collins.Config}\pysiglinewithargsret{\strong{class }\code{collins.}\bfcode{Config}}{\emph{filename}}{}
Bases: \code{object}
\index{imageurl() (collins.Config method)}

\begin{fulllineitems}
\phantomsection\label{collins:collins.Config.imageurl}\pysiglinewithargsret{\bfcode{imageurl}}{\emph{path}, \emph{id}, \emph{width}, \emph{height}, \emph{extension}}{}
\end{fulllineitems}


\end{fulllineitems}

\index{Constants (class in collins)}

\begin{fulllineitems}
\phantomsection\label{collins:collins.Constants}\pysigline{\strong{class }\code{collins.}\bfcode{Constants}}
Bases: \code{object}

Some contsants which are blatantly copied from the php-sdk.
\index{API\_ENVIRONMENT\_LIVE (collins.Constants attribute)}

\begin{fulllineitems}
\phantomsection\label{collins:collins.Constants.API_ENVIRONMENT_LIVE}\pysigline{\bfcode{API\_ENVIRONMENT\_LIVE}\strong{ = `live'}}
\end{fulllineitems}

\index{API\_ENVIRONMENT\_STAGE (collins.Constants attribute)}

\begin{fulllineitems}
\phantomsection\label{collins:collins.Constants.API_ENVIRONMENT_STAGE}\pysigline{\bfcode{API\_ENVIRONMENT\_STAGE}\strong{ = `stage'}}
\end{fulllineitems}

\index{FACET\_BRAND (collins.Constants attribute)}

\begin{fulllineitems}
\phantomsection\label{collins:collins.Constants.FACET_BRAND}\pysigline{\bfcode{FACET\_BRAND}\strong{ = 0}}
\end{fulllineitems}

\index{FACET\_CLOTHING\_MEN\_BELTS\_CM (collins.Constants attribute)}

\begin{fulllineitems}
\phantomsection\label{collins:collins.Constants.FACET_CLOTHING_MEN_BELTS_CM}\pysigline{\bfcode{FACET\_CLOTHING\_MEN\_BELTS\_CM}\strong{ = 190}}
\end{fulllineitems}

\index{FACET\_CLOTHING\_MEN\_DE (collins.Constants attribute)}

\begin{fulllineitems}
\phantomsection\label{collins:collins.Constants.FACET_CLOTHING_MEN_DE}\pysigline{\bfcode{FACET\_CLOTHING\_MEN\_DE}\strong{ = 187}}
\end{fulllineitems}

\index{FACET\_CLOTHING\_MEN\_INCH (collins.Constants attribute)}

\begin{fulllineitems}
\phantomsection\label{collins:collins.Constants.FACET_CLOTHING_MEN_INCH}\pysigline{\bfcode{FACET\_CLOTHING\_MEN\_INCH}\strong{ = 189}}
\end{fulllineitems}

\index{FACET\_CLOTHING\_UNISEX\_INCH (collins.Constants attribute)}

\begin{fulllineitems}
\phantomsection\label{collins:collins.Constants.FACET_CLOTHING_UNISEX_INCH}\pysigline{\bfcode{FACET\_CLOTHING\_UNISEX\_INCH}\strong{ = 174}}
\end{fulllineitems}

\index{FACET\_CLOTHING\_UNISEX\_INT (collins.Constants attribute)}

\begin{fulllineitems}
\phantomsection\label{collins:collins.Constants.FACET_CLOTHING_UNISEX_INT}\pysigline{\bfcode{FACET\_CLOTHING\_UNISEX\_INT}\strong{ = 173}}
\end{fulllineitems}

\index{FACET\_CLOTHING\_UNISEX\_ONESIZE (collins.Constants attribute)}

\begin{fulllineitems}
\phantomsection\label{collins:collins.Constants.FACET_CLOTHING_UNISEX_ONESIZE}\pysigline{\bfcode{FACET\_CLOTHING\_UNISEX\_ONESIZE}\strong{ = 204}}
\end{fulllineitems}

\index{FACET\_CLOTHING\_WOMEN\_BELTS\_CM (collins.Constants attribute)}

\begin{fulllineitems}
\phantomsection\label{collins:collins.Constants.FACET_CLOTHING_WOMEN_BELTS_CM}\pysigline{\bfcode{FACET\_CLOTHING\_WOMEN\_BELTS\_CM}\strong{ = 181}}
\end{fulllineitems}

\index{FACET\_CLOTHING\_WOMEN\_DE (collins.Constants attribute)}

\begin{fulllineitems}
\phantomsection\label{collins:collins.Constants.FACET_CLOTHING_WOMEN_DE}\pysigline{\bfcode{FACET\_CLOTHING\_WOMEN\_DE}\strong{ = 175}}
\end{fulllineitems}

\index{FACET\_CLOTHING\_WOMEN\_INCH (collins.Constants attribute)}

\begin{fulllineitems}
\phantomsection\label{collins:collins.Constants.FACET_CLOTHING_WOMEN_INCH}\pysigline{\bfcode{FACET\_CLOTHING\_WOMEN\_INCH}\strong{ = 180}}
\end{fulllineitems}

\index{FACET\_COLOR (collins.Constants attribute)}

\begin{fulllineitems}
\phantomsection\label{collins:collins.Constants.FACET_COLOR}\pysigline{\bfcode{FACET\_COLOR}\strong{ = 1}}
\end{fulllineitems}

\index{FACET\_CUPSIZE (collins.Constants attribute)}

\begin{fulllineitems}
\phantomsection\label{collins:collins.Constants.FACET_CUPSIZE}\pysigline{\bfcode{FACET\_CUPSIZE}\strong{ = 4}}
\end{fulllineitems}

\index{FACET\_DIMENSION3 (collins.Constants attribute)}

\begin{fulllineitems}
\phantomsection\label{collins:collins.Constants.FACET_DIMENSION3}\pysigline{\bfcode{FACET\_DIMENSION3}\strong{ = 6}}
\end{fulllineitems}

\index{FACET\_GENDERAGE (collins.Constants attribute)}

\begin{fulllineitems}
\phantomsection\label{collins:collins.Constants.FACET_GENDERAGE}\pysigline{\bfcode{FACET\_GENDERAGE}\strong{ = 3}}
\end{fulllineitems}

\index{FACET\_LENGTH (collins.Constants attribute)}

\begin{fulllineitems}
\phantomsection\label{collins:collins.Constants.FACET_LENGTH}\pysigline{\bfcode{FACET\_LENGTH}\strong{ = 5}}
\end{fulllineitems}

\index{FACET\_SHOES\_UNISEX\_ADIDAS\_EUR (collins.Constants attribute)}

\begin{fulllineitems}
\phantomsection\label{collins:collins.Constants.FACET_SHOES_UNISEX_ADIDAS_EUR}\pysigline{\bfcode{FACET\_SHOES\_UNISEX\_ADIDAS\_EUR}\strong{ = 195}}
\end{fulllineitems}

\index{FACET\_SHOES\_UNISEX\_EUR (collins.Constants attribute)}

\begin{fulllineitems}
\phantomsection\label{collins:collins.Constants.FACET_SHOES_UNISEX_EUR}\pysigline{\bfcode{FACET\_SHOES\_UNISEX\_EUR}\strong{ = 194}}
\end{fulllineitems}

\index{FACET\_SIZE (collins.Constants attribute)}

\begin{fulllineitems}
\phantomsection\label{collins:collins.Constants.FACET_SIZE}\pysigline{\bfcode{FACET\_SIZE}\strong{ = 2}}
\end{fulllineitems}

\index{FACET\_SIZE\_CODE (collins.Constants attribute)}

\begin{fulllineitems}
\phantomsection\label{collins:collins.Constants.FACET_SIZE_CODE}\pysigline{\bfcode{FACET\_SIZE\_CODE}\strong{ = 206}}
\end{fulllineitems}

\index{FACET\_SIZE\_RUN (collins.Constants attribute)}

\begin{fulllineitems}
\phantomsection\label{collins:collins.Constants.FACET_SIZE_RUN}\pysigline{\bfcode{FACET\_SIZE\_RUN}\strong{ = 172}}
\end{fulllineitems}

\index{SORT\_CREATED (collins.Constants attribute)}

\begin{fulllineitems}
\phantomsection\label{collins:collins.Constants.SORT_CREATED}\pysigline{\bfcode{SORT\_CREATED}\strong{ = `created\_date'}}
\end{fulllineitems}

\index{SORT\_MOST\_VIEWED (collins.Constants attribute)}

\begin{fulllineitems}
\phantomsection\label{collins:collins.Constants.SORT_MOST_VIEWED}\pysigline{\bfcode{SORT\_MOST\_VIEWED}\strong{ = `most\_viewed'}}
\end{fulllineitems}

\index{SORT\_PRICE (collins.Constants attribute)}

\begin{fulllineitems}
\phantomsection\label{collins:collins.Constants.SORT_PRICE}\pysigline{\bfcode{SORT\_PRICE}\strong{ = `price'}}
\end{fulllineitems}

\index{SORT\_RELEVANCE (collins.Constants attribute)}

\begin{fulllineitems}
\phantomsection\label{collins:collins.Constants.SORT_RELEVANCE}\pysigline{\bfcode{SORT\_RELEVANCE}\strong{ = `relevance'}}
\end{fulllineitems}

\index{SORT\_UPDATED (collins.Constants attribute)}

\begin{fulllineitems}
\phantomsection\label{collins:collins.Constants.SORT_UPDATED}\pysigline{\bfcode{SORT\_UPDATED}\strong{ = `updated\_date'}}
\end{fulllineitems}

\index{TYPE\_CATEGORIES (collins.Constants attribute)}

\begin{fulllineitems}
\phantomsection\label{collins:collins.Constants.TYPE_CATEGORIES}\pysigline{\bfcode{TYPE\_CATEGORIES}\strong{ = `categories'}}
\end{fulllineitems}

\index{TYPE\_PRODUCTS (collins.Constants attribute)}

\begin{fulllineitems}
\phantomsection\label{collins:collins.Constants.TYPE_PRODUCTS}\pysigline{\bfcode{TYPE\_PRODUCTS}\strong{ = `products'}}
\end{fulllineitems}


\end{fulllineitems}

\index{EasyCollins (class in collins)}

\begin{fulllineitems}
\phantomsection\label{collins:collins.EasyCollins}\pysigline{\strong{class }\code{collins.}\bfcode{EasyCollins}}
Bases: \code{object}
\begin{description}
\item[{An easy to use wrapper around the collins api which sets}] \leavevmode
usabilty for anything else.

\end{description}

\end{fulllineitems}

\index{VERSION (in module collins)}

\begin{fulllineitems}
\phantomsection\label{collins:collins.VERSION}\pysigline{\code{collins.}\bfcode{VERSION}\strong{ = `0.0'}}
Version of the python shop SDK.

\end{fulllineitems}



\chapter{Indices and tables}
\label{index:indices-and-tables}\begin{itemize}
\item {} 
\emph{genindex}

\item {} 
\emph{modindex}

\item {} 
\emph{search}

\end{itemize}


\renewcommand{\indexname}{Python Module Index}
\begin{theindex}
\def\bigletter#1{{\Large\sffamily#1}\nopagebreak\vspace{1mm}}
\bigletter{c}
\item {\texttt{collins}}, \pageref{collins:module-collins}
\end{theindex}

\renewcommand{\indexname}{Index}
\printindex
\end{document}
